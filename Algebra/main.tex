\documentclass[journal,12pt,onecolumn]{IEEEtran}
%
\usepackage{setspace}
\usepackage{gensymb}
%\doublespacing
\singlespacing

%\usepackage{graphicx}
%\usepackage{amssymb}
%\usepackage{relsize}
\usepackage[cmex10]{amsmath}
%\usepackage{amsthm}
%\interdisplaylinepenalty=2500
%\savesymbol{iint}
%\usepackage{txfonts}
%\restoresymbol{TXF}{iint}
%\usepackage{wasysym}
\usepackage{amsthm}
%\usepackage{iithtlc}
\usepackage{mathrsfs}
\usepackage{txfonts}
\usepackage{stfloats}
\usepackage{bm}
\usepackage{cite}
\usepackage{cases}
\usepackage{subfig}
%\usepackage{xtab}
\usepackage{longtable}
\usepackage{multirow}
%\usepackage{algorithm}
%\usepackage{algpseudocode}
\usepackage{enumitem}
\usepackage{mathtools}
\usepackage{tikz}
\usepackage{circuitikz}
\usepackage{verbatim}
%\usepackage{tfrupee}
\usepackage[breaklinks=true]{hyperref}
%\usepackage{stmaryrd}
\usepackage{tkz-euclide} % loads  TikZ and tkz-base
\usetkzobj{all}
\usepackage{listings}
    \usepackage{color}                                            %%
    \usepackage{array}                                            %%
    \usepackage{longtable}                                        %%
    \usepackage{calc}                                             %%
    \usepackage{multirow}                                         %%
    \usepackage{hhline}                                           %%
    \usepackage{ifthen}                                           %%
  %optionally (for landscape tables embedded in another document): %%
    \usepackage{lscape}     
\usepackage{multicol}
\usepackage{chngcntr}
%\usepackage{enumerate}

%\usepackage{wasysym}
%\newcounter{MYtempeqncnt}
\DeclareMathOperator*{\Res}{Res}
%\renewcommand{\baselinestretch}{2}
\renewcommand\thesection{\arabic{section}}
\renewcommand\thesubsection{\thesection.\arabic{subsection}}
\renewcommand\thesubsubsection{\thesubsection.\arabic{subsubsection}}

\renewcommand\thesectiondis{\arabic{section}}
\renewcommand\thesubsectiondis{\thesectiondis.\arabic{subsection}}
\renewcommand\thesubsubsectiondis{\thesubsectiondis.\arabic{subsubsection}}

% correct bad hyphenation here
\hyphenation{op-tical net-works semi-conduc-tor}
\def\inputGnumericTable{}                                 %%

\lstset{
%language=C,
frame=single, 
breaklines=true,
columns=fullflexible
}
%\lstset{
%language=tex,
%frame=single, 
%breaklines=true
%}

\begin{document}
%



\newtheorem{theorem}{Theorem}[section]
\newtheorem{problem}{Problem}
\newtheorem{proposition}{Proposition}[section]
\newtheorem{lemma}{Lemma}[section]
\newtheorem{corollary}[theorem]{Corollary}
\newtheorem{example}{Example}[section]
\newtheorem{definition}[problem]{Definition}
%\newtheorem{thm}{Theorem}[section] 
%\newtheorem{defn}[thm]{Definition}
%\newtheorem{algorithm}{Algorithm}[section]
%\newtheorem{cor}{Corollary}
\newcommand{\BEQA}{\begin{eqnarray}}
\newcommand{\EEQA}{\end{eqnarray}}
\newcommand{\define}{\stackrel{\triangle}{=}}

\bibliographystyle{IEEEtran}
%\bibliographystyle{ieeetr}


\providecommand{\mbf}{\mathbf}
\providecommand{\pr}[1]{\ensuremath{\Pr\left(#1\right)}}
\providecommand{\qfunc}[1]{\ensuremath{Q\left(#1\right)}}
\providecommand{\sbrak}[1]{\ensuremath{{}\left[#1\right]}}
\providecommand{\lsbrak}[1]{\ensuremath{{}\left[#1\right.}}
\providecommand{\rsbrak}[1]{\ensuremath{{}\left.#1\right]}}
\providecommand{\brak}[1]{\ensuremath{\left(#1\right)}}
\providecommand{\lbrak}[1]{\ensuremath{\left(#1\right.}}
\providecommand{\rbrak}[1]{\ensuremath{\left.#1\right)}}
\providecommand{\cbrak}[1]{\ensuremath{\left\{#1\right\}}}
\providecommand{\lcbrak}[1]{\ensuremath{\left\{#1\right.}}
\providecommand{\rcbrak}[1]{\ensuremath{\left.#1\right\}}}
\theoremstyle{remark}
\newtheorem{rem}{Remark}
\newcommand{\sgn}{\mathop{\mathrm{sgn}}}
\providecommand{\abs}[1]{\left\vert#1\right\vert}
\providecommand{\res}[1]{\Res\displaylimits_{#1}} 
\providecommand{\norm}[1]{\left\lVert#1\right\rVert}
%\providecommand{\norm}[1]{\lVert#1\rVert}
\providecommand{\mtx}[1]{\mathbf{#1}}
\providecommand{\mean}[1]{E\left[ #1 \right]}
\providecommand{\fourier}{\overset{\mathcal{F}}{ \rightleftharpoons}}
%\providecommand{\hilbert}{\overset{\mathcal{H}}{ \rightleftharpoons}}
\providecommand{\system}{\overset{\mathcal{H}}{ \longleftrightarrow}}
	%\newcommand{\solution}[2]{\textbf{Solution:}{#1}}
\newcommand{\solution}{\noindent \textbf{Solution: }}
\newcommand{\cosec}{\,\text{cosec}\,}
\providecommand{\dec}[2]{\ensuremath{\overset{#1}{\underset{#2}{\gtrless}}}}
\newcommand{\myvec}[1]{\ensuremath{\begin{pmatrix}#1\end{pmatrix}}}
\newcommand{\mydet}[1]{\ensuremath{\begin{vmatrix}#1\end{vmatrix}}}
%\numberwithin{equation}{section}
\numberwithin{equation}{subsection}
%\numberwithin{problem}{section}
%\numberwithin{definition}{section}
\makeatletter
\@addtoreset{figure}{problem}
\makeatother

\let\StandardTheFigure\thefigure
\let\vec\mathbf
%\renewcommand{\thefigure}{\theproblem.\arabic{figure}}
\renewcommand{\thefigure}{\theproblem}
%\setlist[enumerate,1]{before=\renewcommand\theequation{\theenumi.\arabic{equation}}
%\counterwithin{equation}{enumi}


%\renewcommand{\theequation}{\arabic{subsection}.\arabic{equation}}

\def\putbox#1#2#3{\makebox[0in][l]{\makebox[#1][l]{}\raisebox{\baselineskip}[0in][0in]{\raisebox{#2}[0in][0in]{#3}}}}
     \def\rightbox#1{\makebox[0in][r]{#1}}
     \def\centbox#1{\makebox[0in]{#1}}
     \def\topbox#1{\raisebox{-\baselineskip}[0in][0in]{#1}}
     \def\midbox#1{\raisebox{-0.5\baselineskip}[0in][0in]{#1}}

\vspace{3cm}

\title{
%	\logo{
Algebra: Maths Olympiad
%	}
}
\author{ G V V Sharma$^{*}$% <-this % stops a space
	\thanks{*The author is with the Department
		of Electrical Engineering, Indian Institute of Technology, Hyderabad
		502285 India e-mail:  gadepall@iith.ac.in. All content in this manual is released under GNU GPL.  Free and open source.}
	
}	
%\title{
%	\logo{Matrix Analysis through Octave}{\begin{center}\includegraphics[scale=.24]{tlc}\end{center}}{}{HAMDSP}
%}


% paper title
% can use linebreaks \\ within to get better formatting as desired
%\title{Matrix Analysis through Octave}
%
%
% author names and IEEE memberships
% note positions of commas and nonbreaking spaces ( ~ ) LaTeX will not break
% a structure at a ~ so this keeps an author's name from being broken across
% two lines.
% use \thanks{} to gain access to the first footnote area
% a separate \thanks must be used for each paragraph as LaTeX2e's \thanks
% was not built to handle multiple paragraphs
%

%\author{<-this % stops a space
%\thanks{}}
%}
% note the % following the last \IEEEmembership and also \thanks - 
% these prevent an unwanted space from occurring between the last author name
% and the end of the author line. i.e., if you had this:
% 
% \author{....lastname \thanks{...} \thanks{...} }
%                     ^------------^------------^----Do not want these spaces!
%
% a space would be appended to the last name and could cause every name on that
% line to be shifted left slightly. This is one of those "LaTeX things". For
% instance, "\textbf{A} \textbf{B}" will typeset as "A B" not "AB". To get
% "AB" then you have to do: "\textbf{A}\textbf{B}"
% \thanks is no different in this regard, so shield the last } of each \thanks
% that ends a line with a % and do not let a space in before the next \thanks.
% Spaces after \IEEEmembership other than the last one are OK (and needed) as
% you are supposed to have spaces between the names. For what it is worth,
% this is a minor point as most people would not even notice if the said evil
% space somehow managed to creep in.



% The paper headers
%\markboth{Journal of \LaTeX\ Class Files,~Vol.~6, No.~1, January~2007}%
%{Shell \MakeLowercase{\textit{et al.}}: Bare Demo of IEEEtran.cls for Journals}
% The only time the second header will appear i/year/1963s for the odd numbered pages
% after the title page when using the twoside option.
% 
% *** Note that you probably will NOT want to include the author's ***
% *** name in the headers of peer review papers.                   ***
% You can use \ifCLASSOPTIONpeerreview for conditional compilation here if
% you desire.




% If you want to put a publisher's ID mark on the page you can do it like
% this:
%\IEEEpubid{0000--0000/00\$00.00~\copyright~2007 IEEE}
% Remember, if you use this you must call \IEEEpubidadjcol in the second
% column for its text to clear the IEEEpubid ma/year/1963rk.



% make the title area
\maketitle



%\tableofcontents

\bigskip

\renewcommand{\thefigure}{\theenumi}
\renewcommand{\thetable}{\theenumi}
%\renewcommand{\theequation}{\theenumi}

%\begin{abstract}
%%\boldmath
%In this letter, an algorithm for evaluating the exact analytical bit error rate  (BER)  for the piecewise linear (PL) combiner for  multiple relays is presented. Previous results were available only for upto three relays. The algorithm is unique in the sense that  the actual mathematical expressions, that are prohibitively large, need not be explicitly obtained. The diversity gain due to multiple relays is shown through plots of the analytical BER, well supported by simulations. 
%
%\end{abstract}
% IEEEtran.cls defaults to using nonbold math in the Abstract.
% This preserves the distinction between vectors and scalars. However,
% if the journal you are submitting to favors bold math in the abstract,
% then you can use LaTeX's standard command \boldmath at the very start
% of the abstract to achieve this. Many IEEE journals frown on math
% in the abstract anyway.

% Note that keywords are not normally used for peerreview papers.
%\begin{IEEEkeywords}
%Cooperative diversity, decode and forward, piecewise linear
%\end{IEEEkeywords}



% For peer review papers, you can put extra information on the cover
% page as needed:
% \ifCLASSOPTIONpeerreview
% \begin{center} \bfseries EDICS Category: 3-BBND \end{center}
% \fi
%
% For peerreview papers, this IEEEtran command inserts a page break and
% creates the second title. It will be ignored for other modes.
%\IEEEpeerreviewmaketitle


%Download python codes using 
%\begin{lstlisting}
%svn co https://github.com/gadepall/school/trunk/ncert/computation/codes
%\end{lstlisting}

\renewcommand{\theequation}{\theenumi}
\begin{enumerate}[label=\arabic*.,ref=\theenumi]
%\begin{enumerate}[label=\arabic*.,ref=\thesubsection.\theenumi]
\numberwithin{equation}{enumi}
\item m and n are natural numbers with $1 \leq m \leq n.$ In their decimal representations, the last three digits of $1978^{m}$ are equal, respectively, to the last three digits of $1978^{n}.$ Find m and n such that m + n has its least value.

\item Let $a_{k}$(k = 1, 2, 3,....., n,.....) be a sequence of distinct positive integers. Prove that for all natural numbers n,
\begin{align*}
\sum_{k=1}^{n}(\frac{a_k}{k^2}) \geq \sum_{k=1}^{n}(\frac{1}{k})
\end{align*}



\item Two circles in a plane intersect. Let A be one of the points of intersection. Starting simultaneously from A two points move with constant speeds, each point travelling along its own circle in the same sense. The two points return to A simultaneously after one revolution. Prove that there is a fixed point P in the plane such that, at any time, the distances from P to the moving points are equal.

\item Given a plane $\pi$, a point P in this plane and a point Q not in $\pi$, find all points R in $\pi$ such that the ratio $\frac{(QP + PA)}{QR}$ is a maximum.



\item P is a point inside a given triangle ABC.D, E, F are the feet of the perpendiculars from P to the lines BC, CA, AB respectively. Find all P for which
\begin{align*}
\frac{BC}{PD} + \frac{CA}{PE} + \frac{AB}{PF}
\end{align*}
is least.

\item Three congruent circles have a common point O and lie inside a given triangle. Each circle touches a pair of sides of the triangle. Prove that the in-center and the circum center of the triangle and the point O are collinear.


\item The function f(n) is defined for all positive integers n and takes on non-negative integer values. Also, for all m, n\\
f(m + n) - f(m) - f(n) = 0 or 1\\
f(2) = 0, $f(3) > 0,$ and f(9999) = 3333.\\
Determine f(1982).

\item Let S be a square with sides of length 100, and let L be a path within S which does not meet itself and which is composed of line segments $A_0A_1, A_1A_2,......, A_{n - 1}A_n$ with $A_0 \neq A_n.$ Suppose that for every point P of the boundary of S there is a point of L at a distance from P not greater than $\frac{1}{2}.$ Prove that there are two points X and Y in L such that the distance between X and Y is not greater than 1, and the length of that part of L which lies between X and Y is not smaller than 198.


\item Is it possible to choose 1983 distinct positive integers, all less than or equal to $10^{5}$, no three of which are consecutive terms of an arithmetic progression? Justify your answer.

\item Find all functions f defined on the set of positive real numbers which take positive real values and satisfy the conditions:
\begin{enumerate}
\item f(xf(y)) = yf(x) for all positive x, y;
\item f(x) $\rightarrow 0$ as $x \rightarrow \infty$
\end{enumerate}



\item Let d be the sum of the lengths of all the diagonals of a plane convex polygon with n vertices $(n > 3)$, and let p be its perimeter. Prove that
\begin{align*}
n-3 < \frac{2d}{p} < [\frac{n}{2}] [\frac{n + 1}{2}] - 2
\end{align*}
where [x] denotes the greatest integer not exceeding x.








\item Let n and k be given relatively prime natural numbers, $k < n$. Each number in the set M = (1,2,.., $n-1$) is colored either blue or white. It is given that
\begin{enumerate}
\item  for each i $ \in $ M, both i and n - i have the same color;
\item for each i $ \in $ M, i $ \neq $ k, both i and $\begin{vmatrix} i - k \end{vmatrix}$ have the same color. Prove that all numbers in M must have the same color.
\end{enumerate}

\item For any polynomial $P(x) = a_{0} + a_{1} +..... + a_{k}x^{k}$ with integer coefficients, the number of coefficients which are odd is denoted by w(P). For i = 0, 1,....., let $Q_{i}(x) = (1 + x)^{i}.$ Prove that if $i_{1}i_{2},.....,i_{n}$ are integers such that $0 \leq i_{1} < i_{2} < ..... < i_{n},$ then $w(Q_{i1} + Q_{i2}, +  Q_{in}) \geq w(Q_{i1})$.

\item For every real number $x_1,$ construct the sequence $x_1, x_2,$.... by setting
\begin{align*}
x_{n+1} = x_n(x_n + \frac{1}{n})
\end{align*} 
for each $n \geq 1.$
Prove that there exists exactly one value of $x_1$ for which $0 < x_n < x_{n+1} < 1$ for every n.





















\item Let d be any positive integer not equal to 2, 5, or 13. Show that one can find
distinct a, b in the set {2, 5, 13, d} such that ab - 1 is not a perfect square.



















\item In an acute-angled triangle ABC the interior bisector of the angle A intersects BC at L and intersects the circum circle of ABC again at N. From point L perpendiculars are drawn to AB and AC, the feet of these perpendiculars being K and M respectively. Prove that the quadrilateral AKNM and the triangle ABC have equal areas.





















\item Show that set of real numbers x which satisfy the inequality
\begin{align*}
\sum_{k=1}^{70} \frac{k}{x-k} \geq \frac{5}{4}
\end{align*}
is a union of disjoint intervals, the sum of whose lengths is 1988.

\item A function f is defined on the positive integers by
\begin{align*}
f(1) = 1, f(3) = 3,\\
f(2n) = f(n),\\
f(4n + 1) = 2f(2n + 1) - f(n),\\
f(4n + 3) = 3f(2n + 1) - 2f(n)
\end{align*}
for all positive integers n.
Determine the number of positive integers n, less than or equal to 1988, for which f(n) = n.








\item In an acute-angled triangle ABC the internal bisector of angle A meets the circum circle of the triangle again at $A_1$. Points $B_1$ and $C_1$ are defined similarly. Let $A_0$ be the point of intersection of the line A$A_1$ with the external bisectors of angles B and C. Points $B_0$ and $C_0$ are defined similarly. Prove that:
\begin{enumerate}
\item The area of the triangle $A_0B_0C_0$ is twice the area of the hexagon $AC_1BA_1CB_1$.
\item The area of the triangle $A_0B_0C_0$ is at least four times the area of the triangle ABC.
\end{enumerate}

\item Let ABCD be a convex quadrilateral such that the sides AB, AD, BC satisfy AB = AD + BC. There exists a point P inside the quadrilateral at a distance h from the line CD such that AP = h + AD and BP = h + BC. Show that:
\begin{align*}
\frac{1}{\sqrt{h}} \geq \frac{1}{\sqrt{AD}} + \frac{1}{\sqrt{BC}}
\end{align*}























\item Let $n \geq 3$ and consider a set E of 2n - 1 distinct points on a circle. Suppose that exactly k of these points are to be colored black. Such a coloring is "good" if there is at least one pair of black points such that the interior of one of the arcs between them contains exactly n points from E. Find the smallest value of k so that every such coloring of k points of E is good.

\item Let $Q^{+}$ be the set of positive rational numbers. Construct a function f : $Q^{+} \rightarrow Q^{+}$ such that  $Q^{+}$ such that
\begin{align*}
f(xf(y)) = \frac{f(x)}{y}
\end{align*}
for all x, y in $Q^{+}$.







\end{enumerate}

\end{document}


